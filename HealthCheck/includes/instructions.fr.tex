
lien : https://labs.spotify.com/2014/09/16/squad-health-check-model/


\section{Squad Health Check model}

basé sur la version 1, September 2014
Traduction : 
\begin{itemize}
\item Severin Legras
\item Hervé Taboucou
\item Thomas Clavier
\end{itemize}

\subsection{De quoi s'agit-il?}

\begin{itemize}
\item Un atelier et une technique de visualisation aidant les équipes (squads) à s'améliorer.
\end{itemize}

\subsection{Audience}

\begin{itemize}
\item L'équipe elle-même
\item Les personnes apportant leur support à l'équipe (managers, coachs, etc.)
\end{itemize}

\subsection{Comment utiliser ce modèle}

\begin{itemize}
\item Imprimez et plastifiez les cartes


\begin{itemize}
\item Slide 2-5 = Cartes ``Terrifiantes@@@'' (double face)
\item Slide 6-9 = Cartes de vote (double face)
\end{itemize}
\item Rassembler tous les membres de l'équipe dans la même salle
\item Discuter les cartes ``terrifiantes''. Chacune d'entre elle est un indicateur de bonne santé, accompagné d'un exemple de très bonne performance et d'un exemple particulièrement inefficace.
\item tilisant une méthode favorisant les décisions de groupe (par exemple: avec les cartes de vote).


\begin{itemize}
\item Un indicateur ``Vert'' ne signifie pas un état parfaitement idéal, mais que l'équipe est satisfaite sur cet indicateur et ne voit pas d'amélioration significative à mettre en oeuvre dans l'immédiat.
\item Un indicateuer ``jaune'' signifie qu'il y a des problèmes importants qui nécessitent l'attention, mais ils ne constituent pas une situation irrécupérable.
\item Un indicateur ``rouge'' signifie que la situation est critique, et nécessite une amélioration immédiate.
\end{itemize}
\item Faire aussi discuter sur les tendances d'évolution de ces indicateurs (la situation s'améliore-t-elle? Est-elle stable ou se dégrade-t-elle?)
\item Matéraliser visuellement les résultats de ces discussions.
\item Faire utiliser des données quantitatives (estimation, mesures, extrapolation\ldots{}) pour aider l'équipe à s'améliorer.
\item \subsection{Idées de mises en oeuvre}
\item Les cartes sont uniquement un point de départ pour initialiser des converstations productives. L'équipe doit se sentire libre d'ajouter/ôtre/modifier toute question afin de correspondre à ce qu'elle considère comme important pour elle.
\item S'assurer que cet outil est utilisé en support de l'équipe dans son amélioration et surtout pas pour l'évaluers.
\end{itemize}

Le terme ``squad''  (retraduit ici par ``équipe'') est le mot utilisé par Spotify pour une équipe de développement petite, cross-fonctionnelle et auto-organisée.

