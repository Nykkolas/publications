lien : https://labs.spotify.com/2014/09/16/squad-health-check-model/

Traduction en français

\section{Première diapositive}\hypertarget{premire-diapositive}{}\label{premire-diapositive}

\subsection{Squad Health Check model}\hypertarget{squad-health-check-model}{}\label{squad-health-check-model}

version 1, September 2014
Traduction : 
* Severin Legras
* Hervé Taboucou
* Thomas Clavier

\subsubsection{De quoi s'agit-il?}\hypertarget{de-quoi-sagit-il}{}\label{de-quoi-sagit-il}

\begin{itemize}
\item Un atelier et une technique de visualisation aidant les équipes (squads) à s'améliorer.
\end{itemize}

\subsubsection{Audience}\hypertarget{audience}{}\label{audience}

\begin{itemize}
\item L'équipe elle-même
\item Les personnes apportant leur support à l'équipe (managers, coachs, etc.)
\end{itemize}

\subsubsection{Comment utiliser ce modèle}\hypertarget{comment-utiliser-ce-modle}{}\label{comment-utiliser-ce-modle}

\begin{itemize}
\item Imprimez et plastifiez les cartes


\begin{itemize}
\item Slide 2-5 = Cartes ``Terrifiantes@@@'' (double face)
\item Slide 6-9 = Cartes de vote (double face)
\end{itemize}
\item Rassembler tous les membres de l'équipe dans la même salle
\item Discuter les cartes ``terrifiantes''. Chacune d'entre elle est un indicateur de bonne santé, accompagné d'un exemple de très bonne performance et d'un exemple particulièrement inefficace.
\item tilisant une méthode favorisant les décisions de groupe (par exemple: avec les cartes de vote).


\begin{itemize}
\item Un indicateur ``Vert'' ne signifie pas un état parfaitement idéal, mais que l'équipe est satisfaite sur cet indicateur et ne voit pas d'amélioration significative à mettre en oeuvre dans l'immédiat.
\item Un indicateuer ``jaune'' signifie qu'il y a des problèmes importants qui nécessitent l'attention, mais ils ne constituent pas une situation irrécupérable.
\item Un indicateur ``rouge'' signifie que la situation est critique, et nécessite une amélioration immédiate.
\end{itemize}
\item Faire aussi discuter sur les tendances d'évolution de ces indicateurs (la situation s'améliore-t-elle? Est-elle stable ou se dégrade-t-elle?)
\item Matéraliser visuellement les résultats de ces discussions.
\item Faire utiliser des données quantitatives (estimation, mesures, extrapolation\ldots{}) pour aider l'équipe à s'améliorer.
\item \subsection{Idées de mises en oeuvre}\hypertarget{ides-de-mises-en-oeuvre}{}\label{ides-de-mises-en-oeuvre}
\item Les cartes sont uniquement un point de départ pour initialiser des converstations productives. L'équipe doit se sentire libre d'ajouter/ôtre/modifier toute question afin de correspondre à ce qu'elle considère comme important pour elle.
\item S'assurer que cet outil est utilisé en support de l'équipe dans son amélioration et surtout pas pour l'évaluers.
\end{itemize}

Le terme ``squad''  (retraduit ici par ``équipe'') est le mot utilisé par Spotify pour une équipe de développement petite, cross-fonctionnelle et auto-organisée.

\section{Deuxième Slide}\hypertarget{deuxime-slide}{}\label{deuxime-slide}

\subsection{Livraison de valeur}\hypertarget{livraison-de-valeur}{}\label{livraison-de-valeur}

\begin{itemize}
\item Nous livrons des produits extraordinaires@@! Nous en sommes fier et nos clients en sont particulièrement contents.
\item ce que nous livrons est nul. Nous en avons honte. Nos clients nous détestent.
\end{itemize}

\subsection{facilité de livraison}\hypertarget{facilit-de-livraison}{}\label{facilit-de-livraison}

\begin{itemize}
\item Le processus de livraison est simpls, sûrs, indolore et quasiement totalement automatisés.
\item Le processus de livraison est risqué, douloureux, nécessitent beaucoup d'interventions manuelles, qui prennent énorméménet de temps.
\end{itemize}

\subsection{Fun}\hypertarget{fun}{}\label{fun}

\begin{itemize}
\item Nous aimons aller travailler le matin, et avons beaucoup de fun à travailler ensemble.
\item Ennui\ldots{}.
\end{itemize}

\subsection{Bonne santé de la base de code}\hypertarget{bonne-sant-de-la-base-de-code}{}\label{bonne-sant-de-la-base-de-code}

\begin{itemize}
\item Nous sommes fiers de la qualité de notre code. Il est propre, facile à lireet a une grande couverture de tests.
\item Notre code est un tas de fumier et notre dette technique croît de façon incontrôlable.
\end{itemize}

\subsection{Apprentissage}\hypertarget{apprentissage}{}\label{apprentissage}

\begin{itemize}
\item Nous apprenons à chaque instant quelque chose d'intéressant.
\item Nous n'avons jamais le temps d'apprendre quoi que ce soit.
\end{itemize}

\subsection{Mission}\hypertarget{mission}{}\label{mission}

\begin{itemize}
\item Nous savons exactement pourquoi nons somme là, et nous sommes passionnés en connaissance de cause.
\item Nous ne savons pas pouquoi nous sommes là, il n'y a pas de schéma global ou d'orientation globale. Nos ``missions'' sont totalement floues et démotivantes.
\end{itemize}

\subsection{Les acteurs\ldots{} ou les pions\ldots{}}\hypertarget{les-acteurs-ou-les-pions}{}\label{les-acteurs-ou-les-pions}

\begin{itemize}
\item Nous contrôlons notre destin. Nous décidons ce que nous développons et comment nous le développons.
\item Nous ne sommes que des pions sur un échiquier, et nous n'avons aucune influence sur ce que nous construisons et comment nous le développons.
\end{itemize}

\subsection{Vélocité}\hypertarget{vlocit}{}\label{vlocit}

\begin{itemize}
\item Nous faisons immédiatement ce que nous avons décidé. Pas d'attente, pas de délai.
\item Nous ne finissons jamais rien. Nous sommes systématiquement bloqués et interrompus. En particuliern les user stories sont bloqués par les dépendances critiques d'autres produits.
\end{itemize}

\section{Troisième slide}\hypertarget{troisime-slide}{}\label{troisime-slide}

\subsection{Processus adapté}\hypertarget{processus-adapt}{}\label{processus-adapt}

\begin{itemize}
\item Notre façon de travailler nous correspond tout à fait!
\item Notre façon de travailler est ridicule.
\end{itemize}

\subsection{Support}\hypertarget{support}{}\label{support}

\begin{itemize}
\item Nous obentons toujours du support et de l'aide de nos sponsors dès que nous le demandons.
\item Nous sommes régulièrement bloqués car nous ne pouvons obtenir du soutien et de l'aide de la part de nos sponsor quand nous le demandons.
\end{itemize}

\subsection{Travail d'équipe}\hypertarget{travail-dquipe}{}\label{travail-dquipe}

\begin{itemize}
\item Nous sommes un groupe qui a vraiment pris une forme d'équipe, avec un niveau de collaboration extraodinaire.
\item Nous sommes une bande d'individus qui ne se connaissent pas et nous ne sous soucions pas de savoir ce que les autres font.
\end{itemize}
